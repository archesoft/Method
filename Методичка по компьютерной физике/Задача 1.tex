% Преамбула документа
\documentclass[a4paper]{article}
\usepackage[left=2cm,right=2cm,top=2cm,bottom=2cm]{geometry}
\usepackage{amsmath,amssymb,graphicx,textcomp,color,multicol,float}
\usepackage[T2A]{fontenc}
\usepackage[russian]{babel}
%
\begin{document}
	
	% Шапка страницы
	{\centering
		\huge \textbf{Решение задач по физике}\par
		\normalsize
		\itshape Автор: Савченко Н.Е.\par
		\vspace{1ex}
		\hrulefill\par
		\vspace{1em}
	}
	
	\section*{Задача №1}
	
	Колонна мотоциклистов движется по шоссе со скоростью $v = 10\,\text{м/с}$, растянувшись на расстояние $l = 5\,км$. Из хвоста и головы колонны одновременно выезжают навстречу друг другу два мотоциклиста со скоростями $v_1 = 20\,\text{м/с}$ и $v_2 = 15\,\text{м/с}$. Определить время, за которое первый мотоциклист достигнет головы колонны, а второй — её хвоста.
	
	\subsection*{Решение:}
	
	Связываем движущуюся систему отсчета с колонной, принимая за начало координат $O'$ хвост колонны, а за положительное направление оси $O'X'$ — направление движения колонны.
	
	Неподвижную систему отсчета связываем с Землей, совмещаем начало координат $O$ с местом нахождения хвоста колонны в момент выезда мотоциклистов.
	
	Обозначим скорости первого и второго мотоциклистов в движущейся системе отсчета как $\vec{v}'_1$ и $\vec{v}'_2$, соответственно.
	
	По классическому закону сложения скоростей:
	$$
	\vec{v}_1 = \vec{v}'_1 + \vec{u},\quad
	\vec{v}_2 = \vec{v}'_2 + \vec{u},
	$$
	где $\vec{u}$ — скорость колонны.
	
	Отсюда:
	$$
	\vec{v}'_1 = \vec{v}_1 - \vec{u},\quad
	\vec{v}'_2 = \vec{v}_2 - \vec{u}.
	$$
	
	Проекции на ось $O'X'$ равны:
	$$
	v'_1 = v_1 - u,\quad
	v'_2 = v_2 - u.
	$$
	
	Координаты мотоциклистов зависят от времени следующим образом:
	$$
	x_f(t) = (v_1 - u)t,\quad
	x_h(t) = l + ut.
	$$
	
	Время достижения головной части колонны первым мотоциклистом находится из условия:
	$$
	(v_1 - u)t_1 = l,\quad
	t_1 = \frac{l}{v_1 - u}.
	$$
	
	Подставляем значения:
	$$
	t_1 = \frac{5000\,\text{м}}{20\,\text{м/с}-10\,\text{м/с}}=500\,\text{с}.
	$$
	
	Получили, что **первый мотоциклист достигает головы колонны за 500 секунд**.
	
	Аналогичным образом рассчитываем время для второго мотоциклиста:
	$$
	x_t(t)=(v_2+u)t,\quad
	t_2=\frac{l}{v_2+u}.
	$$
	
	Вычислим:
	$$
	t_2=\frac{5000\,\text{м}}{15\,\text{м/с}+10\,\text{м/с}}=200\,\text{с}.
	$$
	
	Итак, **второй мотоциклист достигает хвоста колонны за 200 секунд**.
	
\end{document}