\documentclass[11pt,a4paper]{report}
\usepackage{tcolorbox}
\tcbuselibrary{minted,breakable,xparse,skins}

\definecolor{bg}{gray}{0.95}
\DeclareTCBListing{mintedbox}{O{}m!O{}}{%
  breakable=true,
  listing engine=minted,
  listing only,
  minted language=#2,
  minted style=default,
  minted options={%
    linenos,
    gobble=0,
    breaklines=true,
    breakafter=,,
    fontsize=\small,
    numbersep=8pt,
    #1},
  boxsep=0pt,
  left skip=0pt,
  right skip=0pt,
  left=25pt,
  right=0pt,
  top=3pt,
  bottom=3pt,
  arc=5pt,
  leftrule=0pt,
  rightrule=0pt,
  bottomrule=2pt,
  toprule=2pt,
  colback=bg,
  colframe=orange!70,
  enhanced,
  overlay={%
    \begin{tcbclipinterior}
    \fill[orange!20!white] (frame.south west) rectangle ([xshift=20pt]frame.north west);
    \end{tcbclipinterior}},
  #3}
\begin{document}
\begin{mintedbox}{python}
    mport matplotlib.pyplot as plt
    import numpy as np
    
    # Исходные данные
    v_col = 10                          # м/с (скорость колонны)
    v_mot1 = 20                         # м/с (скорость мотоциклиста 1)
    v_mot2 = 15                         # м/с (скорость мотоциклиста 2)
    L = 5000                            # длина колонны, м
    
    # Максимальное время наблюдений
    t_max = max(L/(v_mot1-v_col), L/(v_mot2+v_col)) * 1.1  # с запасом
    times = np.linspace(0, t_max, 1000)  # массив временных моментов
    
    # Положение хвоста и головы колонны
    pos_tail = v_col * times             # положение хвоста
    pos_head = L + v_col * times         # положение головы
    
    # Положение мотоциклистов:
    # 1-й мотоциклист (от хвоста к голове)
    pos_mot1 = v_mot1 * times           # его координата растет быстрее, чем хвост колонны
    
    # 2-й мотоциклист (от головы к хвосту)
    pos_mot2 = L - v_mot2 * times       # его координата снижается от L
    
    # Временные точки пересечения
    intersect_time1 = L / (v_mot1 - v_col)  # Пересечение мотоциклиста 1 с головой колонны
    intersect_time2 = L / (v_mot2 + v_col)  # Пересечение мотоциклиста 2 с хвостом колонны
    
    # Пересечение двух мотоциклистов
    intersect_time3 = L / (v_mot1 + v_mot2)  # Пересечение мотоциклистов друг с другом
    
    # Выведем полученные результаты
    print("\nРЕЗУЛЬТАТЫ РАСЧЁТА:")
    print("-" * 50)
    print(f"ПЕРЕСЕЧЕНИЯ:")
    print(f"1. Мотоциклист 1 пересекает голову колонны через: {intersect_time1:.2f} секунды.")
    print(f"2. Мотоциклист 2 пересекает хвост колонны через: {intersect_time2:.2f} секунды.")
    print(f"3. Два мотоциклиста встречаются друг с другом через: {intersect_time3:.2f} секунды.")
    
    # Построим график
    plt.figure(figsize=(10, 6))          # увеличим размеры графика
    
    # Основная часть графика
    plt.plot(times, pos_tail, label='Хвост колонны', color='blue', linewidth=2)
    plt.plot(times, pos_head, label='Голова колонны', color='green', linewidth=2)
    plt.plot(times, pos_mot1, label='Мотоциклист 1', color='orange', linewidth=2)
    plt.plot(times, pos_mot2, label='Мотоциклист 2', color='purple', linewidth=2)
    
    # Проводим пунктирные линии с точками пересечения на ось времени
    plt.vlines(intersect_time1, ymin=-100, ymax=np.interp(intersect_time1, times, pos_head), colors='black', linestyles='dashed')
    plt.text(intersect_time1, -200, f'{intersect_time1:.2f}', ha='center', va='bottom', fontweight='bold')
    
    plt.vlines(intersect_time2, ymin=-100, ymax=np.interp(intersect_time2, times, pos_tail), colors='black', linestyles='dashed')
    plt.text(intersect_time2, -200, f'{intersect_time2:.2f}', ha='center', va='bottom', fontweight='bold')
    
    # Третья линия — пересечение мотоциклистов
    plt.vlines(intersect_time3, ymin=-100, ymax=np.interp(intersect_time3, times, pos_mot1), colors='black', linestyles='dashed')
    plt.text(intersect_time3, -200, f'{intersect_time3:.2f}', ha='center', va='bottom', fontweight='bold')
    
    # Оформим график
    plt.xlabel('Время (с)', fontsize=12)
    plt.ylabel('Положение (метры)', fontsize=12)
    plt.title('Движение мотоциклистов и концов колонны', fontsize=14)
    plt.legend(fontsize=10)
    plt.grid(True)
    
    # Масштаб по оси Y увеличим до 11000 метров
    plt.ylim(-100, 11000)
    
    # Отобразим обновлённую версию графика
    plt.show()
\end{mintedbox}

\end{document}